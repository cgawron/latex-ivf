\chapter{Einleitung}

Herzlich willkommen in der \LaTeX{}"=Lehrbrief"=Vorlage. Diese können Sie verwenden, um in \LaTeX{} Lehrbriefe mit den üblichen Formatierungen (wie in der Word"=Vorlage) zu erstellen. Als \LaTeX{}"=Engine wird \XeLaTeX{} verwendet, das evtl.{} nicht in jeder Umgebung installiert ist.

Aus \LaTeX{}"=Sicht ist das Dokument ein Buch (Dokumentklasse ¡scrbook¡), die oberste Hierarchie ist also ¡chapter¡, und Sie können Kapitel in separate Dateien (¡kapitel1.tex¡, ¡kapitel2.tex¡ etc.) auslagern: Das bietet den Vorteil, dass Sie mit ¡\includeonly¡ das Kompilieren auf einzelne Kapitel beschränken können.

Mit ¡\marginnote¡ erzeugen Sie Randnoten im für Lehrbriefe üblichen Stil. Einige Makros erleichtern die Arbeit:

\begin{itemize}
\item ¡\memph{}¡ hebt ein Wort hervor (¡\emph¡) \emph{und} erzeugt eine identische \memph{Randnote}.
\item ¡\imemph{}¡ hebt ein Wort hervor (¡\emph¡) \emph{und} erzeugt eine identische Randnote \emph{und} nimmt das Wort in den \imemph{Index} auf.
\item ¡\iemph{}¡ hebt ein Wort hervor (¡\emph¡) \emph{und} nimmt das Wort in den \iemph{Index} auf.
\item ¡\cmdindex{}¡ fügt einen Befehl ¡command¡\cmdindex{command} in Listing"=Schrift in den Index ein (interessant für Lehrbriefe, in denen es ums Programmieren geht). Zum Auszeichnen von Befehlen im Text können Sie \verb#¡# (das umgedrehte Ausrufezeichen) vor und hinter den Befehl stellen: \verb#¡command¡#\hspace{2mm}. Verwenden Sie nicht ¡\verb#...#¡, da dieser Befehl fehlerhafte Buchstabenabstände erzeugt.
\item ¡\mcmd{}¡ setzt ein Wort in Listing"=Schrift\mcmd{Listing-Randnote} in eine Randnote.
\item ¡\icmd{}¡ nimmt ein Wort via ¡\cmdindex¡ in Listing"=Schrift in den Index auf \emph{und} setzt es in Listing"=Schrift in eine Randnote.
\end{itemize}


Literatur wird wie üblich mit ¡\cite¡ zitiert \cite{esser:gdbs}.


Das folgende Makefile können Sie auf Unix"=ähnlichen Systemen verwenden, um das Dokument zu bauen:

\begin{lstlisting}[language={}]
NAME=lehrbrief-vorlage

all:
	xelatex $(NAME)
	makeindex $(NAME)
	bibtex $(NAME)
	xelatex $(NAME)
\end{lstlisting}

(Die Einrückungen sind Tabulatorzeichen.) Für ein Dokument, das \emph{keine} Literaturverweise (¡\cite¡) enthält, ist der ¡bibtex¡"=Aufruf zu entfernen, weil das Tool sonst eine Fehlermeldung erzeugt.

Abbildung~\ref{fig:beispiel} zeigt eine Beispielabbildung, die in voller Breite eingebaut wurde, Abbildung~\ref{fig:ls-mit-farbe} zeigt eine Abbildung, die auch die Randspalte mitnutzt. Dazu wird innerhalb der ¡figure¡"=Umgebung noch eine ¡wide¡"=Umgebung verwendet.

\begin{figure}[t]
\includegraphics[width=1.0\textwidth]{partitionen}
\caption{Das ist eine Beispielabbildung.}
\label{fig:beispiel}
\end{figure}

\begin{figure}[b!]
\begin{wide}
\includegraphics[width=\textwidth]{ls-mit-farbe}
\caption{Beispiel mit voller Breite: Die Ausgabe von \textzz{ls} ist häufig farbkodiert.}
\label{fig:ls-mit-farbe}
\end{wide}
\end{figure}


Tabelle~\ref{tab:kbyte-mbyte-si} zeigt eine kleine Beispieltabelle.

\begin{table}[b!]
\centering
 \renewcommand{\arraystretch}{1.15}
\hspace{-2.5cm}
\begin{tabular}{|l|l|llr|l|}
\hline
Einheit & Abk. & \multicolumn{3}{l|}{Größe in Byte} & hexadezimal \\
\hline
\hline
Kilobyte (Kibibyte) & KByte & $2^{10} =\!\!\!\!$ & $1024^1 =\!\!\!\!$ & 1 024 & ¡0x00000400¡ \\
\hline
Megabyte (Mebibyte) & MByte & $2^{20} =\!\!\!\!$ & $1024^2 =\!\!\!\!$ & 1 048 576 & ¡0x00100000¡ \\
\hline
Gigabyte (Gibibyte) & MByte & $2^{30} =\!\!\!\!$ & $1024^3 =\!\!\!\!$ & 1 073 741 824 & ¡0x40000000¡ \\
\hline
\hline
Kilobyte (SI) & kB & & $1000^1 =\!\!\!\!$ & 1 000 & ¡0x000003E8¡ \\
\hline
Megabyte (SI) & MB & & $1000^2 =\!\!\!\!$ & 1 000 000 & ¡0x000F4240¡ \\
\hline
Gigabyte (SI) & GB & & $1000^3 =\!\!\!\!$ & 1 000 000 000 & ¡0x3B9ACA00¡ \\
\hline

%Kilobyte (SI)  kB                10001 =                  1 000        0x000003E8
%Megabyte (SI)  MB                10002 =          1 000 000    0x000F4240
%Gigabyte (SI)  GB                10003 =  1 000 000 000        0x3B9ACA00
\end{tabular}
\caption{Klassische Einheiten und SI-Einheiten.}
\label{tab:kbyte-mbyte-si}
\end{table}


Tastenkombinationen können Sie mit ¡\keys¡ erzeugen, z.\,B.

\begin{itemize}
\item \keys{A} (¡\keys{A}¡)
\item \keys{Strg+C} (¡\keys{Strg+C}¡)
\item \keys{Strg+Alt+Entf} (¡\keys{Strg+Alt+Entf}¡)
\end{itemize}


\section{Benötigte Fonts}

Um das gewünschte Layout zu erzeugen, benötigen Sie die richtigen Font"=Dateien. Ich habe das nur unter MacOS und Linux getestet, das Dokument verwendet

\begin{itemize}
\item \iemph{Cambria} (Standard-Font)
\item \iemph{Calibri} (\textsf{Sans-Serif-Font})
\item \iemph{Menlo} (\textzz{Listing-Font})
\end{itemize}
\index{Schriftart!Cambria}%
\index{Schriftart!Calibri}%
\index{Schriftart!Menlo}%

und sollte sich ohne weitere Konfiguration verwenden lassen, wenn diese drei Schriften vorhanden sind. 

Linux"=Anwender können mit folgenden Schritten die Unterstützung für die Fonts aktivieren, die (in dieser Form) unter OpenSuse Leap 42.3 getestet wurden:

\begin{enumerate}
\item Benötigte Pakete mit
\begin{lstlisting}[breaklines=true]
sudo zypper in texlive-latex
sudo zypper in texlive-sidecap texlive-bytefield texlive-mnsymbol texlive-background texlive-comicneue texlive-idxlayout texlive-menukeys texlive-adjmulticol 
\end{lstlisting}
nachinstallieren

\item Dateien ¡Menlo.ttc¡ und ¡Courier New.ttf¡ (z.\,B.{} von einem Mac) besorgen und in den ¡tex¡"=Ordner legen, der das aktuelle Dokument enthält

\item Am Anfang der Datei die Zeile

¡\def\hasMenloFont{True}¡

in

¡\def\hasMenloFont{False}¡

ändern
\end{enumerate}



